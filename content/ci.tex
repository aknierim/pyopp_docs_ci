\section{Continuous Integration (CI)}

\begin{frame}[fragile]{What is Continuous Integration?}
  \begin{itemize}
    \setlength{\itemsep}{1em}
    \item A practice where tests and builds are run automatically after code changes were
      merged
    \item Goal: Find bugs, improve software quality (\eg, performance) and ensure
      your software runs on different platforms
    \item Every commit triggers a CI job
    \item Addressing failed CI jobs before merging a PR ensures code quality
    \item Running tests locally before committing adds an extra layer of ensuring code quality
  \end{itemize}
  \begin{block}{Note}
     The quality of your CI strongly depends on the quality of your tests.
     \begin{itemize}
      \item Requires effort beforehand.
     \end{itemize}
  \end{block}
\end{frame}

\begin{frame}[fragile]{CI Services}
  There are many CI services to choose from. Three widely used services are:
  \begin{description}
    \setlength{\itemsep}{1em}
    \item[Jenkins] Self-hosted, open-source CI service. One of the oldest CI services,
      going back as far as 2005 when it was called Hudson.
    \item[GitLab CI] Widely used in GitLab repos, but also works with
      other services. Can be self-hosted or hosted centrally by GitLab.
    \item[GitHub Actions] A modular CI service developed by Microsoft. Multiple OS
      support and easy to maintain.
  \end{description}
  \begin{itemize}
    \item We will focus on GitLab CI and GitHub Actions (GHA) in the following.
  \end{itemize}
\end{frame}


\begin{frame}[fragile]{
    Getting Started
    \hfill
    \doc{https://docs.github.com/en/actions}{GitHub Actions}
    \doc{https://docs.gitlab.com/ci/}{GitLab CI}
  }
  \begin{columns}[t, onlytextwidth]
    \begin{column}{0.48\textwidth}
      \begin{block}{GitHub Actions}
        \begin{enumerate}
          \item Create a \texttt{ci.yml} file in the \texttt{.github/workflows}
            directory (create the directory if necessary)
          \item Set up some basics in the CI file:
            \footnotesize
            \begin{minted}{yaml}
              name: CI

              on:
                push:
                  branches:
                    - main
                  tags:
                    - '**'
                pull_request:

              env:
                MPLBACKEND: Agg
                PYTEST_ADDOPTS: --color=yes
                GITHUB_PR_NUMBER: ${{ github.event.number }}
            \end{minted}
        \end{enumerate}
      \end{block}
    \end{column}
    \begin{column}{0.48\textwidth}
      \begin{block}{GitLab CI}
        \begin{enumerate}
          \item Create a \texttt{.gitlab-ci.yml} file in the root of your repository
        \end{enumerate}
      \end{block}
    \end{column}
  \end{columns}
\end{frame}
\begin{frame}[fragile]{Getting Started}
  \begin{columns}[t, onlytextwidth]
    \begin{column}{0.48\textwidth}
      \begin{block}{GitHub Actions}
         a
      \end{block}
    \end{column}
    \begin{column}{0.48\textwidth}
      \begin{block}{GitLab CI}
         a
      \end{block}
    \end{column}
  \end{columns}
\end{frame}

