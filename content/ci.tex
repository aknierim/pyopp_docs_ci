\section{Continuous Integration (CI)}

\begin{frame}[fragile]{What is Continuous Integration?}
  \begin{itemize}
    \setlength{\itemsep}{1em}
    \item A practice where tests and builds are run automatically after code changes were
      merged
    \item Goal: Find bugs, improve software quality (\eg, performance) and ensure
      your software runs on different platforms
    \item Every commit triggers a CI job
    \item Addressing failed CI jobs before merging a PR ensures code quality
    \item Running tests locally before committing adds an extra layer of ensuring code quality
  \end{itemize}
  \begin{block}{Note}
     The quality of your CI strongly depends on the quality of your tests.
     \begin{itemize}
      \item Requires effort beforehand.
     \end{itemize}
  \end{block}
\end{frame}

\begin{frame}[fragile]{CI Services}
  There are many CI services to choose from. Three widely used services are:
  \begin{description}
    \setlength{\itemsep}{1em}
    \item[Jenkins] Self-hosted, open-source CI service. One of the oldest CI services,
      going back as far as 2005 when it was called Hudson.
    \item[GitHub Actions] A modular CI service developed by Microsoft. Multiple OS
      support and easy to maintain.
    \item[GitLab CI] Widely used in GitLab repos, but also works with
      other services. Can be self-hosted or hosted centrally by GitLab.
  \end{description}
  \begin{itemize}
    \item We will focus on GitHub Actions (GHA) and GitLab CI in the following.
  \end{itemize}
\end{frame}


\begin{frame}[fragile]{
    Getting Started | GitHub Actions
    \hfill
    \doc{https://docs.github.com/en/actions}{GitHub Actions}
  }
  \begin{enumerate}
    \item Create a \texttt{ci.yml} file in the \texttt{.github/workflows}
        directory (create the directory if necessary)
    \item Set up some basics in the CI file:
      \begin{columns}[t, onlytextwidth]
        \begin{column}{0.52\textwidth}
          \begin{block}{Code}
            \footnotesize
            \begin{minted}{yaml}
              name: CI

              on:
                push:
                  branches:
                    - main
                  tags:
                    - '**'
                pull_request:

              env:
                MPLBACKEND: Agg
                PYTEST_ADDOPTS: --color=yes
            \end{minted}
          \end{block}
        \end{column}
        \begin{column}{0.44\textwidth}
          \pause
          \begin{enumerate}
            \setlength{\itemsep}{1em}
            \item <2-> Name the CI, especially if you are running multiple CIs/CDs
            \item <3-> Set up when the CI should be run, \eg, on every push/merge to \texttt{main}
              and for every PR
            \item <4-> Set up some environment variables, such as the matplotlib backend and color output for pytest
          \end{enumerate}
        \end{column}
      \end{columns}
  \end{enumerate}
\end{frame}

\begin{frame}[fragile]{Getting Started | GitHub Actions}
  \begin{enumerate}
      \setcounter{enumi}{2}
      \item Set up a job for your CI:
        \begin{block}{Code}
          \begin{minted}{yaml}
            jobs:
              tests:  # Name of the job
                runs-on: ubuntu-latest

                defaults:
                  run:
                    # We need login shells (-l) for micromamba to work.
                    shell: bash -leo pipefail {0}

                steps:
                  - uses: actions/checkout@v4
                  - uses: mamba-org/setup-micromamba@v1
                    with:
                      environment-file: environment.yml
          \end{minted}
        \end{block}
    \end{enumerate}
\end{frame}

\begin{frame}[fragile]{Getting Started | GitHub Actions}
  \begin{enumerate}
      \setcounter{enumi}{2}
    \item Set up a job for your CI (cont.):
        \begin{block}{Code}
          \footnotesize
          \begin{minted}{yaml}
                steps:
                  - ...

                  - name: Install dependencies
                    run: |
                      python --version
                      pip install pytest-cov restructuredtext-lint pytest-xdist 'coverage!=6.3.0'
                      pip install -e .[all]
                      pip freeze
                      pip list

                  - name: List installed package versions (conda)
                    if: matrix.environment-type == 'mamba'
                    run: micromamba list

                  - name: Tests
                    run: |
                      pytest -vv --cov --cov-report=xml

                  - name: Upload coverage to Codecov
                    uses: codecov/codecov-action@v4
                    env:
                      CODECOV_TOKEN: ${{ secrets.CODECOV_TOKEN }}
          \end{minted}
        \end{block}
    \end{enumerate}
\end{frame}

\begin{frame}[fragile]{
    Getting Started | GitLab CI
    \hfill
    \doc{https://docs.gitlab.com/ci/}{GitLab CI}
  }
  \begin{columns}[t, onlytextwidth]
    \begin{column}{0.48\textwidth}
      \begin{block}{GitHub Actions}
         a
      \end{block}
    \end{column}
    \begin{column}{0.48\textwidth}
      \begin{block}{GitLab CI}
         a
      \end{block}
    \end{column}
  \end{columns}
\end{frame}

