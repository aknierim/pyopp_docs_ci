\section{Continuous Integration (CI)}

\begin{frame}[fragile]{What You Will Learn:}
  \begin{itemize}
    \item Using CIs to test your code on GitHub or GitLab for multiple platforms
    \item Code coverage
    \item Linting using your CI
    \item Adding badges to your repository
  \end{itemize}
\end{frame}


\begin{frame}[fragile]{What is Continuous Integration?}
  \begin{itemize}
    \setlength{\itemsep}{1em}
    \item A practice where tests and builds are run automatically, \eg, after code changes were
      merged/committed
    \item Goal: Find bugs, improve software quality (\eg, performance) and ensure
      your software runs on different platforms
    \item Every commit triggers a CI job
    \item Addressing failed CI jobs before merging a PR ensures code quality
    \item Running tests locally before committing adds an extra layer of ensuring code quality
  \end{itemize}
  \begin{block}{Note}
     The quality of your CI strongly depends on the quality of your tests.
     \begin{itemize}
      \item Requires effort beforehand.
     \end{itemize}
  \end{block}
\end{frame}

\begin{frame}[fragile]{CI Services}
  There are many CI services to choose from. Three widely used services are:
  \begin{description}
    \setlength{\itemsep}{1em}
    \item[Jenkins] Self-hosted, open-source CI service. One of the oldest CI services,
      going back as far as 2005 when it was called Hudson.
    \item[GitHub Actions] A modular CI service developed by Microsoft. Multiple OS
      support and easy to maintain.
    \item[GitLab CI] Widely used in GitLab repos, but also works with
      other services. Can be self-hosted or hosted centrally by GitLab.
  \end{description}
  \begin{itemize}
    \item We will focus on GitHub Actions (GHA) and GitLab CI in the following.
  \end{itemize}
\end{frame}


\subsection{Getting Started | GitHub Actions}
\begin{frame}[fragile]{
    Getting Started | GitHub Actions
    \hfill
    \doc{https://docs.github.com/en/actions}{GitHub Actions}
  }
  \begin{enumerate}
    \item Create a \texttt{ci.yml} file in the \texttt{.github/workflows}
        directory (create the directory if necessary)
    \item Set up some basics in the CI file:
      \begin{columns}[t, onlytextwidth]
        \begin{column}{0.52\textwidth}
          \begin{block}{Code}
            \footnotesize
            \begin{minted}{yaml}
              name: CI

              on:
                push:
                  branches:
                    - main
                  tags:
                    - '**'
                pull_request:

              env:
                MPLBACKEND: Agg
                PYTEST_ADDOPTS: --color=yes
            \end{minted}
          \end{block}
        \end{column}
        \begin{column}{0.44\textwidth}
          \pause
          \begin{enumerate}
            \setlength{\itemsep}{1em}
            \item <2-> Name the CI, especially if you are running multiple CIs/CDs
            \item <3-> Set up when the CI should be run, \eg, on every push/merge to \texttt{main}
              and for every PR
            \item <4-> Set up some environment variables, such as the matplotlib backend and color output for pytest
          \end{enumerate}
        \end{column}
      \end{columns}
  \end{enumerate}
\end{frame}

\begin{frame}[fragile]{Getting Started | GitHub Actions}
  \begin{enumerate}
      \setcounter{enumi}{2}
      \item Set up a job for your CI that tests your code:
        \begin{block}{Code}
          \begin{minted}{yaml}
            jobs:
              tests:  # Name of the job
                runs-on: ubuntu-latest

                defaults:
                  run:
                    # We need login shells (-l) for micromamba to work.
                    shell: bash -leo pipefail {0}
          \end{minted}
        \end{block}
        \textcolor{vertexLightGrey}{See} \href{https://github.com/mamba-org/setup-micromamba?tab=readme-ov-file#about-login-shells}{About login shells...}
        \textcolor{vertexLightGrey}{why login shells are necessary here.}
    \end{enumerate}
\end{frame}

\begin{frame}[fragile]{
    Getting Started | GitHub Actions
    \hfill
    \doc{https://github.com/actions/checkout}{\texttt{actions/checkout}}
    \doc{https://github.com/mamba-org/setup-micromamba}{\texttt{setup-micromamba}}
  }
  \begin{enumerate}
    \setcounter{enumi}{2}
    \item Set up a job for your CI that tests your code (cont.):
      \begin{block}{Code}
        \footnotesize
        \begin{minted}{yaml}
            tests:
              ...

              steps:
                - uses: actions/checkout@v4

                - uses: mamba-org/setup-micromamba@v1
                  with:
                    environment-file: environment.yml

                - name: Install dependencies
                  run: |
                    python --version
                    pip install -e .[tests]
                    pip freeze
                    pip list

                - name: List installed package versions (conda)
                  run: micromamba list

                - name: Tests
                  run: |
                    pytest -vv --cov --cov-report=xml
        \end{minted}
      \end{block}
  \end{enumerate}
\end{frame}

\subsection{Code Coverage}
\begin{frame}[fragile]{Code Coverage}
  Code coverage shows you how much of your code is covered by tests.
  \begin{itemize}
    \item While the reports produced by pytest in the CI are nice, more elaborate reports are sometimes desireable
    \item Code review services provide such reports
    \item Three commonly used services that are free for FOSS and support multiple languages are:
      \begin{itemize}
        \item Codecov
        \item SonarQube
        \item Codacy
      \end{itemize}
  \end{itemize}
  We will focus on Codecov here, because of its simplicity.
\end{frame}

\begin{frame}[fragile]{
    Codecov
    \hfill
    \doc{https://docs.codecov.com/docs/}{Codecov}
  }
  \begin{enumerate}
    \item Sign up/log in to Codecov, \eg, via GitHub, GitLab, or Bitbucket
    \item Select your repository from your dashoard
    \item Select a setup option, \eg, \enquote{Using GitHub Actions}
    \item Select an upload token. For a single repository, the repository
      token is sufficient
    \item Add the token as repository secret
    \item Update your CI to automatically upload the coverage to Codecov (after the \texttt{Tests} step of your job)
      \begin{block}{Code}
        \footnotesize
        \begin{minted}{yaml}
          - name: Tests
            run: |
              pytest -vv --cov --cov-report=xml

          - name: Upload coverage to Codecov
            uses: codecov/codecov-action@v4
            env:
              CODECOV_TOKEN: ${{ secrets.CODECOV_TOKEN }}
        \end{minted}
      \end{block}
  \end{enumerate}
  \begin{alertblock}{Important}
    \textbf{NEVER} share your token with anyone.
  \end{alertblock}
\end{frame}

\subsection{Multiple Platforms}
\begin{frame}[fragile]{Multiple Platforms | GitHub Actions}
 \begin{itemize}
   \setlength{\itemsep}{1em}
   \item So far we tested only on one platform
   \item We have to ensure our package also runs on all target platforms
      and all target versions of Python
  \end{itemize}
\end{frame}

\begin{frame}[fragile]{Multiple Platforms | GitHub Actions}
  We will modify our CI to test on multiple platforms:
  \begin{enumerate}
    \item The first part remains the same:
      \begin{block}{Code}
        \begin{minted}{yaml}
          name: CI

          on:
            push:
              branches:
                - main
              tags:
                - '**'
            pull_request:

          env:
            MPLBACKEND: Agg
            PYTEST_ADDOPTS: --color=yes
        \end{minted}
      \end{block}
  \end{enumerate}
\end{frame}

\begin{frame}[fragile]{
    Multiple Platforms | GitHub Actions
    \hfill
    \doc{https://docs.github.com/en/actions/writing-workflows/choosing-what-your-workflow-does/running-variations-of-jobs-in-a-workflow}{Matrix Strategies}
  }
  \begin{enumerate}
    \setcounter{enumi}{1}
    \item This time, we will be using GitHub Actions' matrix strategy to define multiple platforms:
      \begin{block}{Code}
        \scriptsize
        \begin{minted}{yaml}
          jobs:
            tests:
              runs-on: ${{ matrix.os }}
              strategy:
                matrix:
                  include:
                    - os: ubuntu-latest
                      python-version: "3.10"
                      install-method: mamba

                    - os: ubuntu-latest
                      python-version: "3.12"
                      install-method: mamba
                      extra-args: ["codecov"]  # lead platform for code cov

                    - os: ubuntu-latest
                      python-version: "3.12"
                      install-method: pip

                    - os: macos-13
                      python-version: "3.10"
                      install-method: pip

              defaults:
                run:
                  # We need login shells (-l) for micromamba to work.
                  shell: bash -leo pipefail {0}
        \end{minted}
      \end{block}
  \end{enumerate}
\end{frame}

\begin{frame}[fragile]{
    Multiple Platforms | GitHub Actions
  }
  \begin{enumerate}
    \setcounter{enumi}{2}
    \item Adding steps:
      \begin{block}{Code}
        \scriptsize
        \begin{minted}{yaml}
            steps:
              - uses: actions/checkout@v4
                with:
                  fetch-depth: 0

              - name: Prepare mamba installation
                if: matrix.install-method == 'mamba' &&  contains(github.event.pull_request.labels.*.name, 'documentation-only') == false
                env:
                  PYTHON_VERSION: ${{ matrix.python-version }}
                run: |
                  # setup correct python version
                  sed -i -e "s/- python=.*/- python=$PYTHON_VERSION/g" environment.yml

              - name: mamba setup
                if: matrix.install-method == 'mamba' && contains(github.event.pull_request.labels.*.name, 'documentation-only') == false
                uses: mamba-org/setup-micromamba@v1
                with:
                  environment-file: environment.yml
                  cache-downloads: true

              - name: Python setup
                if: matrix.install-method == 'pip' && contains(github.event.pull_request.labels.*.name, 'documentation-only') == false
                uses: actions/setup-python@v5
                with:
                  python-version: ${{ matrix.python-version }}
                  check-latest: true
        \end{minted}
      \end{block}
  \end{enumerate}
\end{frame}

\begin{frame}[fragile]{
    Multiple Platforms | GitHub Actions
  }
  \begin{enumerate}
    \setcounter{enumi}{3}
    \item For macOS, we have to fix the Python path:
      \begin{block}{Code}
        \scriptsize
        \begin{minted}{yaml}
          steps:
            - ...

            - if: matrix.install-method == 'pip' && runner.os == 'macOS' && contains(github.event.pull_request.labels.*.name, 'documentation-only') == false
              name: Fix Python PATH on macOS
              run: |
                tee -a ~/.bash_profile <<<'export PATH="$pythonLocation/bin:$PATH"'
        \end{minted}
      \end{block}
  \end{enumerate}
\end{frame}

\begin{frame}[fragile]{
    Multiple Platforms | GitHub Actions
  }
  \begin{enumerate}
    \setcounter{enumi}{4}
    \item Install dependencies and run tests:
      \begin{block}{Code}
        \scriptsize
        \begin{minted}{yaml}
          steps:
            - ...

            - name: Install dependencies
              env:
                PYTHON_VERSION: ${{ matrix.python-version }}
              run: |
                python --version
                pip install -e .[tests]
                pip freeze
                pip list

            - name: List installed package versions (conda)
              if: matrix.environment-type == 'mamba'
              run: micromamba list

            - name: Tests
              run: |
                pytest -vv --cov --cov-report=xml

            - name: Upload coverage to Codecov
              uses: codecov/codecov-action@v4
              env:
                CODECOV_TOKEN: ${{ secrets.CODECOV_TOKEN }}  # make sure you have this set as repository secret
        \end{minted}
      \end{block}
  \end{enumerate}
\end{frame}

\subsection{Linting With the CI}
\begin{frame}[fragile]{Linting With the CI}
  The linting job should preferably be started before the tests.
  \begin{block}{Code}
  \footnotesize
    \begin{minted}{yaml}
      jobs:
        lint:
          runs-on: ubuntu-latest
          steps:
            - uses: actions/checkout@v4
              with:
                fetch-depth: 0

            - uses: actions/setup-python@v5
              with:
                python-version: "3.12"

            - name: Check README syntax
              run: |
                pip install restructuredtext-lint
                restructuredtext-lint README.rst

            - uses: pre-commit/action@v3.0.1
              with:
                extra_args: --files $(git diff origin/main --name-only)
    \end{minted}
  \end{block}
\end{frame}


\subsection{Building the Docs With the CI}
\begin{frame}[fragile]{Building the Docs With the CI}
  The docs job can be started last.
  \begin{block}{Code}
  \footnotesize
    \begin{minted}{yaml}
      jobs:
        docs:
            runs-on: ubuntu-24.04
            steps:
              - uses: actions/checkout@v4
                with:
                  fetch-depth: 0

              - name: Set up Python
                uses: actions/setup-python@v5
                with:
                  python-version: "3.12"

              - name: Install doc dependencies
                run: |
                  sudo apt update -y && sudo apt install -y git build-essential pandoc graphviz ffmpeg
                  pip install -U pip towncrier setuptools
                  pip install -e .[docs]
                  git describe --tags

              - name: Build docs
                run: make -C docs html
    \end{minted}
  \end{block}
\end{frame}


\subsection{Changelog CI}
\begin{frame}[fragile]{Changelog CI}
  \begin{block}{Code}
    \footnotesize
    \begin{minted}{yaml}
      name: Changelog

      on:
        pull_request:
          # should also be re-run when changing labels
          types: [opened, reopened, labeled, unlabeled, synchronize]

      env:
        FRAGMENT_NAME: "docs/changes/${{ github.event.number }}.*.rst"

      jobs:
        changelog:
          runs-on: ubuntu-latest
          steps:
            - uses: actions/checkout@v4
              with:
                fetch-depth: 0

            - name: Check for news fragment
              if: ${{ ! contains( github.event.pull_request.labels.*.name, 'no-changelog-needed')}}
              uses: andstor/file-existence-action@v3
              with:
                files: ${{ env.FRAGMENT_NAME }}
                fail: true
    \end{minted}
  \end{block}
\end{frame}

\subsection{GitLab CI}
\begin{frame}[fragile]{
    GitLab CI
    \hfill
    \doc{https://docs.gitlab.com/ci/}{GitLab CI}
  }
  Let's take what we have written in GitHub Actions to GitLab CI:
  \begin{enumerate}
    \item Create a \texttt{.gitlab-ci.yml} file in the root of your repository
    \item Basic setup:
      \begin{block}{Code}
        \begin{minted}{yaml}
          image: condaforge/miniforge3:24.11.3-0

          variables:
            MPLBACKEND: Agg
            PYTEST_ADDOPTS: --color=yes

          stages:
            - lint
            - test
            - docs
        \end{minted}
      \end{block}
  \end{enumerate}
\end{frame}

\begin{frame}[fragile]{GitLab CI}
  \begin{enumerate}
    \setcounter{enumi}{2}
    \item Create a reusable template for all platforms:
      \begin{block}{Code}
        \footnotesize
        \begin{minted}{yaml}
          .test_template: &test_template
            stage: test
            before_script:
              - apt-get update && apt-get install -y curl bzip2

              - mamba env create -f environment.yml
              - source /opt/conda/etc/profile.d/conda.sh
              - conda activate [your_env_name]

              - pip install pytest-cov restructuredtext-lint pytest-xdist 'coverage!=6.3.0'
              - pip install -e .
              - pip freeze
              - pip list
            script:
              - pytest -vv --cov --cov-report=xml
            artifacts:
              paths:
                - coverage.xml
              reports:
                coverage_report:
                  coverage_format: cobertura
                  path: coverage.xml
        \end{minted}
      \end{block}
  \end{enumerate}
\end{frame}


\begin{frame}[fragile]{
    GitLab CI
    \hfill
    \doc{https://docs.gitlab.com/ci/variables/predefined_variables/}{Predefined Variables}
  }
  \begin{enumerate}
    \setcounter{enumi}{3}
    \item Add jobs for each platform:
      \begin{block}{Code}
        \footnotesize
        \begin{minted}{yaml}
          test:python-3.10:
            <<: *test_template
            variables:
              PYTHON_VERSION: "3.10"
            rules:
              - if: $CI_PIPELINE_SOURCE == "merge_request_event" && $CI_MERGE_REQUEST_LABELS =~ /documentation-only/
                when: never
              - when: always

          test:python-3.11:
            <<: *test_template
            variables:
              PYTHON_VERSION: "3.11"
            after_script:
              - curl -Os https://cli.codecov.io/latest/linux/codecov
              - chmod +x codecov
              - ./codecov upload-process -t $CODECOV_TOKEN
            rules:
              - if: $CI_PIPELINE_SOURCE == "merge_request_event" && $CI_MERGE_REQUEST_LABELS =~ /documentation-only/
                when: never
              - when: always
        \end{minted}
      \end{block}
  \end{enumerate}
\end{frame}


\begin{frame}[fragile]{Linting | GitLab CI}
  \begin{block}{Code}
    \footnotesize
    \begin{minted}{yaml}
      lint:
        stage: lint
        image: python:3.12-slim
        before_script:
          - apt update -y
          - apt install -y git
          - git fetch --unshallow || true

          - pip install restructuredtext-lint pre-commit
          - pre-commit install
        script:
          - restructuredtext-lint README.rst
          - pre-commit run --files $(git diff origin/main --name-only) || true
        rules:
          - when: always
      \end{minted}
   \end{block}
\end{frame}


\begin{frame}[fragile]{Test Docs Build | GitLab CI}
  \begin{block}{Code}
    \footnotesize
    \begin{minted}{yaml}
      docs:
        stage: test
        image: python:3.12-slim
        before_script:
          - apt update -y
          - apt install -y git build-essential pandoc graphviz ffmpeg make

          - pip install -U pip towncrier setuptools
          - pip install -e .[docs]

          - git describe --tags
        script:
          - make -C docs html
        rules:
          - when: always
      \end{minted}
   \end{block}
\end{frame}


\begin{frame}[fragile]{Changelog | GitLab CI}
  \begin{block}{Code}
    \scriptsize
    \begin{minted}{yaml}
      changelog:
        stage: test
        image: ubuntu:latest
        rules:
          - if: $CI_PIPELINE_SOURCE == "merge_request_event"
        variables:
          FRAGMENT_NAME: "docs/changes/${CI_MERGE_REQUEST_IID}.*.rst"
        before_script:
          - apt-get update && apt-get install -y git
        script:
          - git fetch --unshallow || true
          
          - |
            if echo "$CI_MERGE_REQUEST_LABELS" | grep -q "no-changelog-needed"; then
              echo "Skipping changelog check due to 'no-changelog-needed' label"
              exit 0
            fi
          
          - |
            if ls $FRAGMENT_NAME 1> /dev/null 2>&1; then
              echo "Changelog fragment found: $(ls $FRAGMENT_NAME)"
            else
              echo "Error: No changelog fragment found matching pattern: $FRAGMENT_NAME"
              echo "Please add a changelog fragment for this merge request."
              exit 1
            fi
        only:
          - merge_requests
     \end{minted}
   \end{block}
\end{frame}


\subsection{Badges}
\begin{frame}[fragile]{Badges}
  \begin{itemize}
    \item Badges are small images that display the status of your CI or code coverage
    \item Including a badge in your README is as simple as adding an image:
      \begin{block}{Code}
        \scriptsize
        \begin{minted}{rst}
        =============================
        package |ci| |codecov| |pypi|
        =============================

        .. |ci| image:: https://github.com/your_project/package/actions/workflows/ci.yml/badge.svg?branch=main
           :target: https://github.com/your_project/package/actions/workflows/ci.yml?branch=main
           :alt: Test Status

        .. Badges from GitLab CI
        .. |ci| image:: https://gitlab.com/your_project/package/badges/-/pipeline.svg
           :target: https://gitlab.com/your_project/package/-/pipelines/latest
           :alt: Test Status

        .. |codecov| image:: https://codecov.io/github/your_project/package/badge.svg
           :target: https://codecov.io/github/your_project/package
           :alt: Code coverage

        .. |pypi| image:: https://badge.fury.io/py/package.svg
           :target: https://pypi.org/project/package
           :alt: PyPI release
        \end{minted}
      \end{block}
    \item Great sources for creating badges (\eg, PyPI version): \href{https://badge.fury.io/}{badge.fury.io}
      and \href{https://shields.io}{shields.io}
  \end{itemize}
\end{frame}

