\section{Documentation}


\begin{frame}[fragile]{What You Will Learn:}
  \begin{itemize}
    \item Documenting your code using \emph{Sphinx}
    \item \emph{reStructuredText} (\reST/\texttt{RST}) syntax
    \item Using \emph{Read the Docs}
  \end{itemize}
\end{frame}


\begin{frame}[fragile]{Why Should We Document Our Code?}
  Well-documented code improves...
  \begin{itemize}
    \item Maintainability: Future Developers, debugging, collaboration, ...
    \item Accessibility: Make your package easier to understand for new users
    \item Collaboration: Docs as a shared knowledge source
  \end{itemize}
\end{frame}


\begin{frame}[fragile]{
  What is Sphinx?
  \hfill
  \doc{https://www.sphinx-doc.org}{Sphinx}}
  \begin{itemize}
    \item Open-source, extensible documentation generator written in Python
    \item Multiple output formats: \texttt{HTML}, \LaTeX{} (for \texttt{PDF}), \texttt{ePub}, and more...
    \item Creates cross-references within your project and across different projects
    \item Allows documentation using a mark-up language (\reST)
    \item Supports various docstring formats (some through extensions)
  \end{itemize}
\end{frame}


\begin{frame}[fragile]{Installation}
  Sphinx can be installed via standard package managers:
  \begin{itemize}
    \item Installing from PyPI using \texttt{pip}:
    \begin{minted}{shell-session}
      $ pip install -U sphinx
    \end{minted}
    \item Conda/Mamba:
    \begin{minted}{shell-session}
      $ mamba install sphinx
    \end{minted}
    \begin{minted}{shell-session}
      $ conda install -c conda-forge sphinx
    \end{minted}

    \item Debian/Ubuntu using \texttt{apt}:
    \begin{minted}{shell-session}
      $ sudo apt install python3-sphinx
    \end{minted}

    \item Homebrew:
      \begin{minted}{shell-session}
        $ brew install sphinx-doc
      \end{minted}
  \end{itemize}
\end{frame}

\begin{frame}[fragile]{Getting Started}
  \begin{minted}[escapeinside=||]{shell-session}
    $ sphinx-quickstart docs
    |\textcolor{cmauve}{> Separate source and build directories (y/n) [n]:}| y
    |\textcolor{cmauve}{> Project name:}| ...
    |\textcolor{cmauve}{> Author name(s):}| ...
    |\textcolor{cmauve}{> Project release []:}| ...
  \end{minted}

  \xdefinecolor{dircolor}{HTML}{f0c481}
  \forestset{
    textfile/.style = {
      execute at begin node=\textcolor{vertexDarkRed}{\faFile*}\space
    },
    batchfile/.style = {
      execute at begin node=\textcolor{vertexDarkRed}{\faTerminal}\space
    },
    makefile/.style = {
      execute at begin node=\textcolor{vertexDarkRed}{\faFile}\space
    },
    pythonfile/.style = {
      execute at begin node=\textcolor{vertexDarkRed}{\faPython}\space
    },
    opened/.style = {execute at begin node=\textcolor{dircolor}{\faFolderOpen}\space},
    closed/.style = {execute at begin node=\textcolor{dircolor}{\faFolder}\space},
    dir tree/.style = {
      grow'=0,
      font=\ttfamily,
      folder,
      fit=band,
      s sep=5pt,
      before computing xy={l=20pt},
      edge={rounded corners=2pt},
    }
  }

  \begin{center}
  \pause
  \begin{minipage}{.45\textwidth}
  \begin{forest}
    for tree={dir tree}
    [docs, opened
      [build, closed]
      [source, opened
        [\_static, closed]
        [\_templates, closed]
        [conf.py, pythonfile]
        [index.rst, textfile]
      ]
      [make.bat, batchfile]
      [Makefile, makefile]
    ]
  \end{forest}
  \end{minipage}
  \hfill
  \pause
  \begin{minipage}{.45\textwidth}
    \begin{forest}
      for tree={dir tree}
      [docs, opened
        [\_build, closed]
        [\_static, closed]
        [\_templates, closed]
        [conf.py, pythonfile]
        [index.rst, textfile]
        [make.bat, batchfile]
        [Makefile, makefile]
      ]
    \end{forest}
  \end{minipage}
  \end{center}
\end{frame}
