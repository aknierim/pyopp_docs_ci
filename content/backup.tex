\appendix
\section{Backup}

\begin{frame}{Sphinx Extensions: Notable Mentions}
  \begin{itemize}
    \setlength{\itemsep}{1em}
    \item \href{https://nbsphinx.readthedocs.io}{\texttt{nbsphinx}}: Built on \texttt{nbconvert} to include Jupyter Notebooks in the docs
    \item \href{https://sphinx-design.readthedocs.io/en/latest/}{\texttt{sphinx-design}}: Adds components for responsive web components
    \item \href{https://sphinx-gallery.github.io/stable/index.html}{\texttt{sphinx-gallery}}: Creates example galleries from python scripts.
  \end{itemize}
\end{frame}

\begin{frame}[fragile]{
    Jupyter Notebooks in Sphinx
    \hfill
    \doc{https://nbsphinx.readthedocs.io/}{\texttt{nbsphinx}}
  }
  \begin{itemize}
    \setlength{\itemsep}{0.75em}
    \item Sometimes it is nicer to write a tutorial in a Jupyter notebook
    \item \texttt{nbsphinx} is a Sphinx extension that will include notebooks in your docs:
    \begin{minted}{shell-session}
      $ mamba install nbsphinx
      $ pip install nbsphinx
    \end{minted}
    \item Add \texttt{nbsphinx} to your extensions list in \texttt{conf.py}
    \begin{block}{Code}
      \scriptsize
      \begin{minted}{python}
        extensions = [
            ...
            "nbsphinx"
        ]
      \end{minted}
    \end{block}
    \item Create a new directory, \eg, \texttt{tutorials}, move the notebook (\eg, \texttt{plot.ipynb}) there,
    and add an \texttt{index.rst}:
    \begin{block}{Code}
      \scriptsize
      \begin{minted}{rst}
        *********
        Tutorials
        *********

        .. toctree::
           :maxdepth: 1
           :glob:

           plot .. name of the notebook
      \end{minted}
    \end{block}
  \end{itemize}
\end{frame}

